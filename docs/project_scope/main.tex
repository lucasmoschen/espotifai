\documentclass{article}
\usepackage[utf8]{inputenc}
\usepackage{hyperref}

\title{Espotifai\\
\large Automatic Playlist Recommender}
\author{Lucas Emanuel Resck Domingues\\
Lucas Machado Moschen\\
\textit{School of Applied Mathematics}\\
\textit{Getulio Vargas Foundation}\\}

\date{July 2020}

\begin{document}

\maketitle

\section{Project statement}

    \textbf{Question}: Which song should we recommend based on
    a playlist and user/context information?

    We can define playlist as a sequence of tracks (audio recordings).
    In this project we aim to study the playlist generation problem, that is,
    given a pool of tracks, a background knowledge about the user,
    and some metadata of songs and playlists, the goal is to create a sequence
    of tracks that satisfies some target as best as possible.

    Our work can be followed on our \href{https://github.com/lucasresck/espotifai}{GitHub repository}.

    
\section{Preliminary exploratory data analysis}

    \begin{enumerate}
        \item \textbf{Inspect the available datasets.}
        
            The main dataset for this especific work should be The Million
            Playlist Dataset from RecSys Challenge 2018, but it's unavailable.
            So we searched for another datasets and API's in order to get data.
            If we use API's, we will clean and treat the data.
        
            The options are:

            \begin{itemize}
                \item Spotify API with SpotiPy library to create a dataset;
                \item Last.fm API to create dataset;
                \item \href{http://millionsongdataset.com/}{Million Song Dataset}
                (subset with 10 thousand songs);
                \item \href{https://dbis.uibk.ac.at/node/263}{NowPlaying}:
                Dataset obtained with information from Twitter, when a tweet is tagged with \#nowplaying, \#listento or \#listeningto.
                \item \href{https://github.com/felipevieira/computacao-e-musica-lsd/blob/master/sbcm-2017/Datasets/MPSD%20v1.0.csv}{Million Playlist Song Dataset}
                (Subset with 10 thousand songs);
                \item \href{https://github.com/mdeff/fma}{FMA:
                A Dataset For Music Analysis Data Set};
                \item - Others datasets.
            \end{itemize}

        \item \textbf{Analyse the available features of the datasets and how it
        can help us to solve our question.}
        
            We imagine some interesting features that can help us in this work:

            \begin{itemize}
                \item Music genre;
                \item Location;
                \item Time of the day;
                \item etc.
            \end{itemize}

        \item{\textbf{Calculate some statistics about data.}}
        
            \begin{itemize}
                \item Percentage of listened genres;
                \item Percentage of genders;
                \item Percentege of countrys;
                \item Count of missing data;
                \item Distribution of artists auditions.
                \item etc.
            \end{itemize}

        \item{\textbf{Explore visually the influence of some categorical
        and quantitative variables on the features of the audios.}}
        
            For example:

            \begin{itemize}
                \item How age influences the musical genres heard?
                \item How the country influences it?
                \item How gender affects it?
                \item etc.
            \end{itemize}

        \end{enumerate}

\end{document}